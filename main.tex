\documentclass{article}
\usepackage{graphicx} %package to manage images
\graphicspath{ {images/} }
\usepackage[rightcaption]{sidecap}
\usepackage{wrapfig}
\begin{document}

% The fancyvrb package lets us customize the formatting of verbatim
% environments.  We use a slightly smaller font.

\title{Decay lab \\ Physics SL 12} % Title
\author{Jie Ji} % Author name
\begin{document}
\maketitle % Insert the title, author and date
\large

\maketitle

\section{Introduction}
Decay constant is where the amount of radioactive atoms decrease over a period of time. There are formulas that is associate with this matter. This experiment will be searching for the rates of decay for unstable elements that can be measured. To look for the exact time that the nucleus will decay, even though the number might be extremely large to find the rate of decay. 


\section{Apparatus}

$$E=\frac{hc}{\lambda}$$

This is the formuale for calculating the energy released in one single photon. Where h is a Constance, c is the speed of light, and lamda is the wavelength. 

$$A=Aoe^{-kt}$$

A is population equal to 0, A0 is where population after time period. K is exponential growth rate, and t is amount of time for half-life. 

\newpage
\section{Experiment}
Basically we are going to use a JAVA program to begin our experiment. We use this simulator on the computer to find out the possibilities of decays of certain amount of atom inside. The amount of atoms im going to use in this experiment is 10000, the rate of possible decay of A to B 0.5. And the rate of decay from B to C is 0.1 

\section{conclusion}
Based on the graph, we can clearly sees the smooth line in graph of A B and C. It uses 87 steps to finish the decay, the software only uses the possibilities to calculating the steps. it is all hypothesis and possible of decay. Every time that enter the exact number will not get the same result at all. The following pages will be giving detailed tables of the decay for the plot of graph.

\section{Data}

\begin{figure}[htp]
\centering
\includegraphics[width=8cm, height=9cm]{graph}
\caption{Bohr's picture}
\label{fig:lion}
\end{figure}

\newpage

\begin{table}[htbp]
\begin{center}
\large
\begin{tabular}{lllll}

Step & A   & B & C  \\                                                      

0 & 10000 & 0 & 0 \\
1 & 5047 & 4953 & 0 \\
2 & 2511 & 7021 & 468 \\
3 & 1240 & 7591 &1169\\
4 & 625 & 7466 & 1909\\
5 & 314 & 7010 & 2676\\
6 & 156 & 6474 & 3370\\
7 & 78 & 5888 & 4034\\
8 & 44 & 5320 & 4636\\
9 & 20 & 4847 & 5133\\
10 & 9 & 4382 & 5609\\
11 & 7 & 3937 & 6056\\
12 & 2 & 3565 & 6433\\
13 & 1 & 3221 & 6778\\
14 & 0 & 2908 & 7092\\
15 & 0 & 2628 & 7372\\
16 & 0 & 2349 & 7651\\
17 & 0 & 2112 & 7888\\
18 & 0 & 1874 & 8126\\
19 & 0 & 1699 & 8301\\
20 & 0 & 1529 & 8471\\
21 & 0 & 1379 & 8621\\
22 & 0 & 1246 & 8754\\
23 & 0 & 1124 & 8876\\
24 & 0 & 1008 & 8992\\
25 & 0 & 914 & 9086\\
26 & 0 & 810 & 9190\\
27 & 0 & 731 & 9269\\
28 & 0 & 649 & 9351\\
29 & 0 & 574 & 9426\\
30 & 0 & 507 & 9493\\
31 & 0 & 455 & 9545\\
32 & 0 & 410 & 9590\\
33 & 0 & 378 & 9622\\
34 & 0 & 347 & 9653\\
35 & 0 & 314 & 9686\\
    

\end{tabular}
\end{center}
  \caption{Experimental data 1-35 steps}
  \label{tab:font-sizes}
\end{table}


\begin{table}[htbp]
\begin{center}
\large
\begin{tabular}{lllll}

36 & 0 & 282 & 9718\\                                                     




37 & 0 & 248 & 9752\\
38 & 0 & 225 & 9775\\
39 & 0 & 209 & 9791\\
40 & 0 & 187 & 9813\\
41 & 0 & 166 & 9834\\
42 & 0 & 152 & 9848\\
43 & 0 & 130 & 9870\\
44 & 0 & 116 & 9884\\
45 & 0 & 107 & 9893\\
46 & 0 & 93 & 9907\\
47 & 0 & 84 & 9916\\
48 & 0 & 74 & 9926\\
49 & 0 & 67 & 9933\\
50 & 0 & 58 & 9942\\
51 & 0 & 52 & 9948\\
52 & 0 & 49 & 9951\\
53 & 0 & 43 & 9957\\
54 & 0 & 38 & 9962\\
55 & 0 & 33 & 9967\\
56 & 0 & 30 & 9970\\
57 & 0 & 28 & 9972\\
58 & 0 & 28 & 9972\\
59 & 0 & 25 & 9975\\
60 & 0 & 20 & 9980\\
61 & 0 & 16 & 9984\\
62 & 0 & 15 & 9985\\
63 & 0 & 15 & 9985\\
64 & 0 & 14 & 9986\\
65 & 0 & 11 & 9989\\
66 & 0 & 10 & 9990\\
67 & 0 & 9 & 9991\\
68 & 0 & 9 & 9991\\
69 & 0 & 9 & 9991\\
70 & 0 & 9 & 9991\\

\end{tabular}
\end{center}
  \caption{Experimental data 36-75}
  \label{tab:font-sizes}
\end{table}

\begin{table}[htbp]
\begin{center}
\large
\begin{tabular}{lllll}

71 & 0 & 8 & 9992\\                                                   



72 & 0 & 7 & 9993\\
73 & 0 & 6 & 9994\\
74 & 0 & 5 & 9995\\
75 & 0 & 4 & 9996\\
76 & 0 & 4 & 9996\\
77 & 0 & 3 & 9997\\
78 & 0 & 3 & 9997\\
79 & 0 & 3 & 9997\\
80 & 0 & 1 & 9999\\
81 & 0 & 1 & 9999\\
82 & 0 & 1 & 9999\\
83 & 0 & 1 & 9999\\
84 & 0 & 1 & 9999\\
85 & 0 & 1 & 9999\\
86 & 0 & 1 & 9999\\
87 & 0 & 1 & 9999\\

\end{tabular}
\end{center}
  \caption{Experimental data 71-87}
  \label{tab:font-sizes}
\end{table}




\end{document}